%%%%%%%%%%%%%%%%%%%%%%%%%%%%%%%%%%%%%%%%%%%%%%%%%%%%%%%%%%%%%%%%%%%%%%%%%%%%%%%%
%2345678901234567890123456789012345678901234567890123456789012345678901234567890
%        1         2         3         4         5         6         7         8

\documentclass[letterpaper, 10 pt, conference]{ieeeconf}  % Comment this line out if you need a4paper

%\documentclass[a4paper, 10pt, conference]{ieeeconf}      % Use this line for a4 paper

\IEEEoverridecommandlockouts                              % This command is only needed if 
                                                          % you want to use the \thanks command

% \overrideIEEEmargins                                      % Needed to meet printer requirements.

%In case you encounter the following error:
%Error 1010 The PDF file may be corrupt (unable to open PDF file) OR
%Error 1000 An error occurred while parsing a contents stream. Unable to analyze the PDF file.
%This is a known problem with pdfLaTeX conversion filter. The file cannot be opened with acrobat reader
%Please use one of the alternatives below to circumvent this error by uncommenting one or the other
%\pdfobjcompresslevel=0
%\pdfminorversion=4

% See the \addtolength command later in the file to balance the column lengths
% on the last page of the document

% The following packages can be found on http:\\www.ctan.org
%\usepackage{graphics} % for pdf, bitmapped graphics files
%\usepackage{epsfig} % for postscript graphics files
%\usepackage{mathptmx} % assumes new font selection scheme installed
%\usepackage{times} % assumes new font selection scheme installed
%\usepackage{amsmath} % assumes amsmath package installed
%\usepackage{amssymb}  % assumes amsmath package installed
\usepackage[english]{babel}
% Set page size and margins
% Replace `letterpaper' with `a4paper' for UK/EU standard size
\usepackage[letterpaper,top=2cm,bottom=2cm,left=3cm,right=3cm,marginparwidth=1.75cm]{geometry}
% Useful packages
\usepackage{amsmath}
\usepackage{graphicx}
\usepackage[colorlinks=true, allcolors=blue]{hyperref}
\usepackage{float}
\usepackage{subcaption}
\title{\LARGE \bf Multi agent planning to detect obstacles and navigate in an unknown environment}


\author{Group Names}%



\begin{document}



\maketitle
\thispagestyle{empty}
\pagestyle{empty}


%%%%%%%%%%%%%%%%%%%%%%%%%%%%%%%%%%%%%%%%%%%%%%%%%%%%%%%%%%%%%%%%%%%%%%%%%%%%%%%%
\begin{abstract}
 Rescue operations for individuals lost in a forest require drones to quickly scan a large unknown environment and quickly provide relief. We propose to work on this problem by breaking it into 2 stages. 1) multiple drones cooperatively scan the entire area with unknown obstacles communincating with each other to create a single map. 2) Once a person is found, another drone can be sent from the base to their location to provide relief by finding an optimal path.
\end{abstract}

%%%%%%%%%%%%%%%%%%%%%%%%%%%%%%%%%%%%%%%%%%%%%%%%%%%%%%%%%%%%%%%%%%%%%%%%%%%%%%%%
\section{INTRODUCTION}
Objective:
\begin{enumerate}
    \item Multiple drones plan and scan the unknown environment. The unknown environment is broken down into multiple grids and the task of the drones is to visit each grid in a cooperative manner.
    \item The drones detect obstacles around them which are then transformed to the initial frame and updated on a common map.
    \item When a person is found, a different drone using the common map calculates the optimal path to reach their location and provide relief.
\end{enumerate}
Environment:
\begin{enumerate}
    \item We will use PyBulletDrone environment for this project where cylinders are used to represent trees.
    \item The environment spawns these trees at random locations throughout the map which is unknown to the planning algorithm. These obstacles are sensed by the robot as they come within the sensing range of onboard sensors.
    \item A person is said to be detected when the drone is close enough to their location.
\end{enumerate}
Procedure:
\begin{enumerate}
    \item Our workspace is \(R^3\). Configuration space of the quadcopter is \(R^3 \times SO(3)\). We will plan in the workspace \(R^3\) itself. \textcolor{red}{Should we search for the optimal path in the 3D World space or the 6D C-Space of the robot or 12D Position + Velocity space? We can use kinodynamics of the quadrotor to simplify the problem.}
    \item We use a high level planner that directs the drones to scan the environment in a Breadh first search manner. The robots communicate with a central node that directs the search such that 2 drones do not explore the same region.
    \item The drone gets its next target location from the high level planner. We use the \(RRT^*\) algorithm on the common map to plan the path for the drone from its current location to the target location.
    \item We use the \(RRT^*\) algorithm over \(A^*\) because the entire map is not known. As \(RRT^*\) builds the tree incrementally, we can update the path with new information without recomputing the map entirely.
    \item Then make the quadcopters follow this path using an off the shelf control algorithm.
    \item The algorithm will be evaluated on the time taken to find the lost person and the the time taken to provide aid to the person once found.
\end{enumerate}

\section{ROBOT MODEL}
% 1/2-1 page
 
We have decided to use Quadrotor as our robot. Quadrotor's workspace and configuration space are \(R^3\) and  \(R^3 \times SO(3)\), respectively.
The derivation of the dynamics is as follows\cite{6289431}.
\subsection{Model of a rotor}

Each rotor rotates with angular velocity $\omega$ and generates a lift force F and moment M. Moment is acting opposite to the directing of rotation.

The lift Force F and moment M of ith rotor can be calculated by:

$F_i = k_f * \omega_i^2$, \hspace{1cm} $k_f = k_T*\rho * D^4$

$M_i = k_m * \omega_i^2$, \hspace{1cm} $k_m = k_Q*\rho * D^5$

where:

%$k_T$ is thrust coefficient

%$k_Q$ is torque 

%$rho$ is fluid density

%$D$ is diameter of propeller

\subsection{Equations of Motion}

Total thrust and moment is the sum of individual ones in each of the 4 rotors.

Thrust: $F = \sum F_i - mga_3$

%Here, $F_x$ are individual lift forces by the propellers and $m*g$ is the one by gravity.

Moment: $M = \sum r_i*F_i + \sum M_i$

%Here, $r_i*F_i$ are the moments created by forces in quadrotor's centre of gravity and $M_x$ are the individual moments created by the propellers.


\subsection{Newton-Euler Equations for Quadrotor}

\textit{Linear Dynamics}:

Applying Newton's Second Law for system of particles, we get (in inertial frame);

$F = m * a$ 

%$acceleration (\ddot{r}) = d\dot{r}/dt$, where $\dot{r} = [u,v,w]^T$ (3.3)
In matrix form, we get;

\[
m*\ddot{r} = \begin{bmatrix}
    0 \\
    0 \\
    -m*g
\end{bmatrix} + 
R_\psi\phi\theta \begin{bmatrix}
    0 \\
    0 \\
    \sum F_i
\end{bmatrix}
\]
\\

\textit{Rotational Dynamics}:

%Applying Euler's rotation equations, we get (in body frame);

%$M_c = ^AdH_c^B/dt = ^BdH_c^B/dt + ^A\omega^B \times H_c^B$

%where, $H_c$ is the angular momentum and $^A\omega^B$ is angular velocity of body B in frame A which is given by $p.b_1 + q.b_2 + r.b_3$

Applying General vector form of Euler's equation;
$M_c = I\dot{\omega} + \omega \times (I\omega)$

For Quadrotor, after rearranging the general vector form, we get;
\[
I\begin{bmatrix}
    \dot{p} \\
    \dot{q} \\
    \dot{r}
\end{bmatrix} = 
\begin{bmatrix}
    L(F_2 - F_4) \\
    L(F_3 - F_1) \\
    M_1 - M_2 + M_3 - M_4
\end{bmatrix} - 
\begin{bmatrix}
    p \\
    q \\
    r
\end{bmatrix}
\times I\begin{bmatrix}
    p \\
    q \\
    r
\end{bmatrix}
\]

Let $\gamma = k_M/k_F$, $M_i = \gamma F_i$, we get;

\[
I\begin{bmatrix}
    \dot{p} \\
    \dot{q} \\
    \dot{r}
\end{bmatrix} = 
\begin{bmatrix}
    0 & L & 0 & -L \\
    -L & 0 & L & 0 \\
    \gamma & -\gamma & \gamma & -\gamma
\end{bmatrix}
\begin{bmatrix}
    F_1 \\
    F_2 \\
    F_3 \\
    F_4
\end{bmatrix}- 
\begin{bmatrix}
    p \\
    q \\
    r
\end{bmatrix}
\times I\begin{bmatrix}
    p \\
    q \\
    r
\end{bmatrix}
\]

Final equations using Linear and Rotational dynamics equations, we get;

\[
\begin{bmatrix}
    T \\
    \tau_1 \\
    \tau_2 \\
    \tau_3
\end{bmatrix} = 
\begin{bmatrix}
    k_F & k_F & k_F & k_F \\
    0 & Lk_F & 0 & -Lk_F \\
    -Lk_F & 0 & Lk_F & 0 \\
    k_M & -k_M & k_M & -k_M
\end{bmatrix}
\begin{bmatrix}
    \omega_1^2 \\
    \omega_2^2 \\
    \omega_3^2 \\
    \omega_4^2
\end{bmatrix}
\]

%TODO shorten, narrative, separate in paragraphs

%\subsection{State Variables}

%$u, \phi, p$ are the linear velocity along the roll-axis direction, angle rotated and the angular velocity about the roll-axis.
%$w, \psi, r$ is the linear velocity along the yaw-axis direction, angle rotated and the angular velocity about the yaw-axis.
%$v, \theta, q$ is the linear velocity along the pitch-axis direction, angle rotated and the angular velocity about the pitch-axis.
%$x, y, z$ are the position of quadrotor along $a_1, a_2, a_3$ axis of the inertial frame. 

%\begin{figure}[H]
%    \centering
%    \includegraphics[width=0.5\linewidth]{variables.png}
%    \caption{Inertial frame a and body frame b}
%    \label{fig:enter-label}
%\end{figure}


%\subsection{Rotation Matrices}

%We have five coordinate reference frames in total. Namely, Inertial frame, vehicle frame, yaw-adjusted frame, pitch adjusted frame and body frame. Inertial frame is fixed on the ground at a predefined home location ($a_1, a_2, a_3$).Vehicle frame has axes parallel to the inertial frame but has the origin shifted to the quadrotor’s center of gravity. Vehicle frame’s yaw is adjusted to match the quadrotor’s yaw to get the yaw-adjusted frame which is then pitch adjusted to get pitch-adjusted frame. Finally body frame is obtained by adjusting the roll of the pitch adjusted frame. The transformation from the inertial to vehicle frame is just as simple translation. The transformation from vehicle to body frame ($b_1, b_2, b_3$) is given by the following rotation matrix:
%\vspace{10pt}

%$R_v^b(\phi, \theta, \psi) = R_p^b(\phi)R_y^p(\theta)R_v^y(\psi)$

%\[
%= \begin{bmatrix}
%    1 & 0 & 0 \\
%    0 & $cos\phi$ & $sin\phi$ \\
%    0 & $-sin\phi$ & $cos\phi$ \\
%\end{bmatrix}.
%\begin{bmatrix}
%    $cos\theta$ & 0 & $-sin\theta$ \\
%    0 & 1 & 0 \\
%    $sin\theta$ & 0 & $cos\theta$
%\end{bmatrix}
%\begin{bmatrix}
    
%\]

%\subsection{Kinematics}

%\subsubsection{Translational Kinematics}
%The state variables(\(\dot{x}\), \(\dot{y}\), \(\dot{z}\)) are inertial frame parameters whereas, velocities ($u, v, w$) are body frame parameters. They can be related through the transformation matrix as follows:

%\[
%\begin{bmatrix}
%  \dot{x} \\
%  \dot{y} \\
%  \dot{z}
%\end{bmatrix} = 
%(R_v^b)^T \cdot \begin{bmatrix}
%    $u$ \\
%    $v$ \\
%    $w$     
%\end{bmatrix}
%\]

%\subsubsection{Rotational Kinematics}
%Since the yaw(relative to vehicle frame), pitch(relative to yaw-adjusted frame) and roll(relative to pitch-adjusted frame) are measured relative to different coordinate systems, the transformation for each is different. So, the angular velocities (p,q,r) are obtained as follows:

%\[
%\begin{bmatrix}
%  p \\
%  q \\
% r
%\end{bmatrix} = 
%\begin{bmatrix}
%    \dot{\phi} \\
%    0 \\
%    0     
%\end{bmatrix} + 
%R_p^b(\phi) \cdot \begin{bmatrix}
%    0 \\
%    \dot{\theta} \\
%    0     
%\end{bmatrix} +
%R_p^b(\phi) \cdot R_y^p(\theta) \cdot \begin{bmatrix}
%    0 \\
%    0 \\
%    \dot{\psi}     
%\end{bmatrix}
%= \begin{bmatrix}
%    1 & 0 & -sin\theta \\
%    0 & cos\theta & sin\phi cos\theta \\
%    0 & -sin\phi & cos\phi cos\theta
%\end{bmatrix} 
%\cdot \begin{bmatrix}
%    \dot{\phi} \\
%    \dot{\theta} \\
%    \dot{\psi}
%\end{bmatrix}
%\]

\section{MOTION PLANNING}
 1/2-1 page

\section{RESULTS}
1 page

\section{DISCUSSION}
1/2-1 page
 

\addtolength{\textheight}{-12cm}   % This command serves to balance the column lengths
                                  % on the last page of the document manually. It shortens
                                  % the textheight of the last page by a suitable amount.
                                  % This command does not take effect until the next page
                                  % so it should come on the page before the last. Make
                                  % sure that you do not shorten the textheight too much.

%%%%%%%%%%%%%%%%%%%%%%%%%%%%%%%%%%%%%%%%%%%%%%%%%%%%%%%%%%%%%%%%%%%%%%%%%%%%%%%%



%%%%%%%%%%%%%%%%%%%%%%%%%%%%%%%%%%%%%%%%%%%%%%%%%%%%%%%%%%%%%%%%%%%%%%%%%%%%%%%%



%%%%%%%%%%%%%%%%%%%%%%%%%%%%%%%%%%%%%%%%%%%%%%%%%%%%%%%%%%%%%%%%%%%%%%%%%%%%%%%%



\begin{thebibliography}{99}

\bibitem{c1} Example, Example bibliography, The bibliography: A great example, 1st ed. vol. 1, Delft, 2019, pp. 1-2


\end{thebibliography}




\end{document}
