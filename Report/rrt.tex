%! Author = shantnavagarwal
%! Date = 08/01/24
\subsection{RRT Star}
We use the RRT Star algorithm first described by Karaman et al.\cite{karaman2011sampling} for planning an error free path between the start and goal positions.
The drones maintain a constant and unique altitude at all times ensuring they don't collide with each other.
The drones are constrained to maintain a constant yaw angle as this does not have any influence on the simulation in our chosen scenario.
\subsubsection{Planning}
Therefore, planning is done in the X-Y 2 Dimensional space to reduce search space and improve performance.
The start position is the drone's current position, goal position is received from the global planner.
The occupancy map is sliced to contain the start and goal locations with sufficient padding.
This reduces search space for $RRT^*$ algorithm and improves performance.
\subsubsection{Algorithm \& Execution}
We have implemented the RRT* algorithm from scratch in Python.
The $RRT^*$ takes start, goal position; occupancy map and radius as input parameters.
Occupancy map contains all the valid positions in XY space that the drone can visit.
We calculate this occupancy map by convolving the drone's occupancy map over the world map (received from the simulation).
All vertices in RRT* including start and goal are stored as Node(s).
The Node object stores their position, cost (cost = parent's cost + Euclidean distance from parent), parent nodes and child nodes.
All vertices and edges in the RRT* are stored in a networkx\cite{SciPyProceedings_11} undirected graph G.
%\begin{algorithm}
%\caption{$RRT^*$ Algorithm}\label{alg:cap}
\begin{algorithmic}[1]
    \State \textbf{Do for n iterations}
    \State Node: $x_{rand}$ = A random point sampled from the free space (uniform distribution)
    \State Node: $x_{nearest}$ = Point nearest to $x_{rand}$
    \State Node: $x_{new}$ = Point in free space along line (and farthest to) $x_{nearest}$ to $x_{rand}$ with distance $\leq$ radius
    \State Node(s): $x_{arr}$ = List of nodes with Euclidean distance from $x_{new}$ $\leq$ radius
    \State Find node $x_{min}$ \in $x_{arr}$ such that $x_{min}$ cost + Euclidean distance between $x_{min}$ and $x_{new}$ is minimum.
    \State Add node $x_{new}$ and edge ($x_{min}$, $x_{arr}$) to G. Set $x_{min}$ as the parent of $x_{arr}$
    \For{$x_{near}$ in $x_{arr}$}
    \State Bool: collision = True if line between $x_{near}$ and $x_{arr}$ goes through an obstacle else False
    \State Float: $new\_cost$ = $x_{near}$ cost + Euclidean distance between $x_{near}$ and $x_{arr}$
    \State \textbf{if}(collision == False $\&$ $new\_cost$ $<$ $x_{near}$ cost)
    \State\hspace{\algorithmicindent} Remove edge (parent $x_{near}$, $x_{near}$)
    \State\hspace{\algorithmicindent} Add edge ($x_{new}$, $x_{near}$) and set $x_{new}$ as parent of $x_{near}$
    \State \textbf{end if}
    \EndFor
    \State After every 100 iterations check if goal is within the radius of a vertice and can be connected to it without collision
\end{algorithmic}
%\end{algorithm}
The algorithm terminates once the Goal can be connected.
\subsubsection{Path Following}
Path is found by connecting the parent node of all nodes starting with the goal node.
While flying, at each timestep, the drone finds all the vertices in the path that are within a distance radius.
Then the drone flies towards the vertex that is furthest up the path.
